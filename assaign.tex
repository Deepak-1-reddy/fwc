\documentclass[10pt,-letter paper]{article}
\usepackage[left=1in, right=0.75in, top=1in, bottom=0.75in]{geometry}
\usepackage{graphicx} % Required for inserting images
\usepackage{siunitx}
\usepackage{setspace}
\usepackage{gensymb}
\usepackage{xcolor}
\usepackage{caption}
%\usepackage{subcaption}
\doublespacing
\singlespacing
\usepackage[none]{hyphenat}
\usepackage{amssymb}
\usepackage{relsize}
\usepackage[cmex10]{amsmath}
\usepackage{mathtools}
\usepackage{amsmath}
\usepackage{commath}
\usepackage{amsthm}
\interdisplaylinepenalty=2500
%\savesymbol{iint}
\usepackage{txfonts}
%\restoresymbol{TXF}{iint}
\usepackage{wasysym}
\usepackage{amsthm}
\usepackage{mathrsfs}
\usepackage{txfonts}
\let\vec\mathbf{}
\usepackage{stfloats}
\usepackage{float}
\usepackage{cite}
\usepackage{cases}
\usepackage{subfig}
%\usepackage{xtab}
\usepackage{longtable}
\usepackage{multirow}
%\usepackage{algorithm}
\usepackage{amssymb}
%\usepackage{algpseudocode}
\usepackage{enumitem}
\usepackage{mathtools}
%\usepackage{eenrc}
%\usepackage[framemethod=tikz]{mdframed}
\usepackage{listings}
%\usepackage{listings}
\usepackage[latin1]{inputenc}
%%\usepackage{color}{   
%%\usepackage{lscape}
\usepackage{textcomp}
\usepackage{titling}
\usepackage{hyperref}
%\usepackage{fulbigskip}   
\usepackage{tikz}
\usepackage{graphicx}
\lstset{
  frame=single,
  breaklines=true
}
\let\vec\mathbf{}
\usepackage{enumitem}
\usepackage{graphicx}
\usepackage{siunitx}
\let\vec\mathbf{}
\usepackage{enumitem}
\usepackage{graphicx}
\usepackage{enumitem}
\usepackage{tfrupee}
\usepackage{amsmath}
\usepackage{amssymb}
\usepackage{mwe} % for blindtext and example-image-a in example
\usepackage{wrapfig}
\graphicspath{{figs/}}
\providecommand{\cbrak}[1]{\ensuremath{\left\{#1\right\}}}
\providecommand{\brak}[1]{\ensuremath{\left(#1\right)}}
\newcommand{\sgn}{\mathop{\mathrm{sgn}}}
\providecommand{\abs}[1]{\left\vert#1\right\vert}
\providecommand{\res}[1]{\Res\displaylimits_{#1}} 
\providecommand{\norm}[1]{\left\lVert#1\right\rVert}
%\providecommand{\norm}[1]{\lVert#1\rVert}
\providecommand{\mtx}[1]{\mathbf{#1}}
\providecommand{\mean}[1]{E\left[ #1 \right]}
\providecommand{\fourier}{\overset{\mathcal{F}}{ \rightleftharpoons}}
%\providecommand{\hilbert}{\overset{\mathcal{H}}{ \rightleftharpoons}}
\providecommand{\system}{\overset{\mathcal{H}}{ \longleftrightarrow}}
 %\newcommand{\solution}[2]{\textbf{Solution:}{#1}}
%\newcommand{\solution}{\noindent \textbf{Solution: }}
\newcommand{\cosec}{\,\text{cosec}\,}
\providecommand{\dec}[2]{\ensuremath{\overset{#1}{\underset{#2}{\gtrless}}}}
\newcommand{\myvec}[1]{\ensuremath{\begin{pmatrix}#1\end{pmatrix}}}
\newcommand{\myaugvec}[2]{\ensuremath{\begin{amatrix}{#1}#2\end{amatrix}}}
\newcommand{\mydet}[1]{\ensuremath{\begin{vmatrix}#1\end{vmatrix}}}
\title{MATHEMATICS}
\author{SECTION A}
\date{\today}
\begin{document}

\maketitle

\begin{enumerate}
\section{Linear Forms}
	\item  If $a*b$ denotes the larger of '$a$' and '$b$' and if $a o b = \brak{a * b} + 3$, then write the value of $\brak{5} o \brak{10}$, where $*$ and $o$ are binary operations.
 \item Find the equations of the tangent and the normal,to the curve $16x^{2}+9y^{2}=145$ at the point $\brak{x_1,y_1}$,where $x_1=2$ and $y_1>0$.
 \item Using the integration, find the area of the region in the first quadrant enclosed by the X-axis, the line $y=x$ and the circle $x^{2}+y^{2}=32$.


 \section{Vectors}

	\item Find the magnitude of each of the vectors $\overrightarrow{\mathbf{a}}$ and $\overrightarrow{\mathbf{b}}$, having the same magnitude such that the angle between them is ${60\degree}$ and their scalar product is $\frac{9}{2}$.
 
 \item Find the shortest distance between the lines $\overrightarrow{\mathbf{r}}=\brak{4\hat{i}-\hat{j}}+\lambda\brak{\hat{i}+2\hat{j}-3\hat{k}}$ and $\overrightarrow{\mathbf{r}}=\brak{\hat{i}-\hat{j}+2\hat{k}}+\mu\brak{2\hat{i}+4\hat{j}-5\hat{k}-} $.

\item Show that the function $f:\mathbb{R}\rightarrow \mathbb{R}$ defined by $f\brak{x} = \frac{x}{x^{2}+1}, \forall \mathbf{x}\in \mathbb{R}$ is neither one-one nor onto. Also,if $g:\mathbb{R} \rightarrow \mathbb{R}$ is defined as $g\brak{x}=2x-1$,find $fog\brak{x}$.

\item If $\theta$ is the angle between the two vectors $\hat{i}-2\hat{j}+3\hat{k}$ and $3\hat{i}-2\hat{j}+\hat{k}$, find $\sin\theta$.

\item  Find the distance of the point \brak{-1,-5,-10} from the point of intersection of the line $\overrightarrow{\mathbf{r}}=2\hat{i}-\hat{j}+2\hat{k} + \lambda\brak{3\hat{i}+4\hat{j}+2\hat{k}}$ and the plane $\overrightarrow{\mathbf{r}}\cdot\brak{\hat{i}-\hat{j}+\hat{k}}=5$.

\item Let $\overrightarrow{\mathbf{a}}=4\hat{1}+5\hat{j}-\hat{k}$,$\overrightarrow{\mathbf{b}}=\hat{i}-4\hat{j}+5\hat{k}$ and $\overrightarrow{\mathbf{c}}=3\hat{i}+\hat{j}-\hat{k}$. find a vector $\overrightarrow{\mathbf{d}}$ which is perpendicular to both $\overrightarrow{\mathbf{c}}$ and $\overrightarrow{\mathbf{b}}$ and $\overrightarrow{\mathbf{d}}\cdot \overrightarrow{\mathbf{a}}=21$

\section{Matrices}

\item  If the matrix $A = \myvec{ 0 & a & -3 \\ 2 & 0 & -1 \\ b & 1 & 0 }$ is skew symmetric, find the values of '$a$' and '$b$'.

\item Using properties of determinants,prove that 
	\begin{align*}
		\mydet{1 & 1 & 1+3x \\ 1+3y & 1 & 1 \\ 1 & 1+3z & 1 }=9\brak{3xyz+xy+yz+zx}
	\end{align*}
\item If $A=\myvec{2 & -3 & 5 \\ 3 & 2 & -4 \\ 1 & 1 & -2}$, Find the $A^{-1}$. Use it ton solve the system of equations 
	\begin{align*}
		2x-3y+5z=11 \\
		3x-2y-4z=-5\\
		x+y-2z=-3
	\end{align*}
\item Using elementary row transformations, find the inverse of the matrix $A=\myvec{1 & 2 & 3 \\ 2 & 5 & 7 \\ -2 & -4 & -5}$.
\item Given $A$ = $\myvec{ 2 & -3 \\ -4 & 7 }$, compute $A^{-1}$ and show that $2A^{-1} = 9I-A$.
\section{Algebra}

\item Find the value of
	\begin{align*}
		\tan^{-1}\sqrt{3}-\cot^{-1}\brak{\sqrt{-3}}
	\end{align*}
 \item Prove that 
			\begin{align*}
		3\sin^{-1}x=\sin^{-1}\brak{3x-4x^3}, x\in\brak{\frac{-1}{2},\frac{1}{2}}
			\end{align*} 
   \section{Differentiation}
 
 \item Differentiate 
	\begin{align*}
	 \tan^{-1}\brak{\frac{1+\cos x}{\sin x}} 
	\end{align*} 
		with respect to $x$.
  

\item Find the differential equation representing the family of curves $y=ae^{bx+5}$, where $a$ and $b$ are arbitrary constants.

\item If $y=\sin\brak{\sin x}$, prove that 
		\begin{align*}
			\frac{d^{2}y}{dx^{2}}+\tan x \frac{dy}{dx}+y\cos^{2}x=0
		\end{align*}
	\item Find the particular solution of of the differential equation $e^{x}\tan y~ dx +\brak{2-e^{x}}\sec^{2} y ~dy=0$, given that $y=\frac{\pi}{4}$ when $x=0$. 
	\item Find the particular solution of the differential equation  $\frac{dy}{dx}+2y\tan x =\sin x$, given that $y=0$ when $x=\frac{\pi}{3}$.
 
 \item If ${\brak{x^{2}+y^{2}}}^{2}=xy$, find $\frac{dy}{dx}$

\item If $x=a\brak{20-\sin 20}$ and $y=a\brak{1-\cos 20}$, find $\frac{dy}{dx}$ when $\theta=\frac{\pi}{3}$

\section{Integration}
\item Evaluate: 
\begin{align*}
	\int_{0}^{\frac{\pi}{4}}\frac{\sin x+\cos x}{16+9 \sin 2x}.1
\end{align*}
\item Evaluate 
	\begin{align*}
		\int_{1}^{3} \brak{x^{2}+3x+e^{x}}dx,
	\end{align*}
	as the limit of the sum.
\item Evaluate:
			\begin{align*}
			\int\frac{\cos 2x+2\sin^{2}x}{\cos^{2}x}dx
				\end{align*}
\item Find:
	\begin{align*}
		\int\frac{2\cos x}{\brak{1-\sin x}\brak{1+\sin^{2}x}}dx
	\end{align*}
 
 \section{Probability}
\item The total cost $C\brak{x}$ associated with the production of $x$ units of an item is given by $C\brak{x} = {0.005x}^3-{0.02x}^2+30x+5000$. Find the material cost when $3$ units are produced, where by marginal cost we mean the instantaneous rate of change of total cost at any level of output.

		
\item A black and a red die are rolled together. Find the conditional probability of obtaining the sum $8$, given that the red die resulted in a number less than $4$.
\item Two numbers are selected at random \brak{without\hspace{2pt}replacement} from the first five positive integers. Let $X$ denote the larger of the two numbers obtained. Find the mean and variance of $X$.
\item  Suppose a girl throws a die. If she gets $1$ or $2$, she tosses a coin three times and notes the number of tails. If she gets $3$, $4$, $5$ or $6$, she tosses a coin once and notes whether a 'head' or 'tail' is obtained. If she obtained exactly one 'tail', what is the probability that she threw $3$, $4$, $5$ or $6$ with the die ? 
\section{optimization}
\item An open tank with a square base and vertical sides is to be constructed from a metal sheet so as to hold a given quantity of water. Show that the cost of material will be least when depth of the tank is half of its width. If the cost is to be borne by nearby settled lower income families, for whom water will be provided, what kind of value is hidden in this question ?

\item A factory manufactures two types of screws $A$ and $B$, each type requiring the use of two machines, an automatic and a hand-operated. It takes $4$ minutes on the automatic and $6$ minutes on the hand-operated machines to manufacture a packet of screws '$A$' while it takes $6$ minutes on the automatic and $3$ minutes on the hand-operated machine to manufacture a packet of screws '$B$'. Each machine is available for at most $4$ hours on any day. The manufacturer can sell a packet of screws '$A$' at a profit of $70$ paise and screws '$B$' at a profit of$\rupee 1$. Assuming that he can sell all the screws he manufactures, how many packets of each type should the factory owner produce in a day in order to maximize his profit ? Formulate the above LPP and solve it graphically and find the maximum profit.

\section{Functions}
\item Find the intervals in which the function $f\brak{x}=\frac{x^4}{4}-x^{3}-5x^{2}+24x+12 $ is 
\begin{enumerate}
  \item strictly increasing.
 \item strictly decreasing.
\end{enumerate}
 \item Let $A=\{x \in Z : 0 \leq x \leq 12\}$. Show that $R=\{\brak{a,b}: a,b \in A,\vert{a-b}$ is divisible by $4$\} is equivalence relation.find the set of all elements related to 1. also write the equivalence class $\brak{2}$.


\end{enumerate}
\end{document}
